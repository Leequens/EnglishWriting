\documentclass[10pt,twocolumn,letterpaper]{article}
\usepackage{cvpr}
\usepackage{enumerate}
\usepackage{times}
\usepackage{epsfig}
\usepackage{float}
\usepackage{graphicx}
\usepackage{amsmath}
\usepackage{amssymb}
\usepackage{multirow}
\usepackage{array}
\usepackage{cite}
\usepackage{calc}
\usepackage[pagebackref=true,breaklinks=true,letterpaper=true,colorlinks,bookmarks=false]{hyperref}
\title{\textbf{Natural Language Object Retrieval}}
\author{Zhang, Liqiang\\\\June 14, 2018}
\cvprfinalcopy
\def\cvprPaperID{****}
\def\httilde{\mbox{\tt\raisebox{-.5ex}{\symbol{126}}}}
\begin{document}
\maketitle
\par
\begin{abstract}
  In this paper, the author address the task of natural language object retrieval, to localize a target object within a given image based on a natural language query of the object. Natural language object retrieval differs from text-based image retrieval task as it involves spatial information about objects within the scene and global scene context. To address this issue, the auhtor propose a novel Spatial Context Recurrent ConvNet (SCRC) model as scoring function on candidate boxes for object retrieval, integrating spatial configurations and global scene-level contextual information into the network. The model processes query text, local image descriptors, spatial configurations and global context features through a recurrent network, outputs the probability of the query text conditioned on each candidate box as a score for the box, and can transfer visual-linguistic knowledge from image captioning domain to our task.
\end{abstract}
\section{Introduction}
Significant progress has been made in object detection in recent years; with the help of Convolutional Neural Networks (CNNs), it is possible to detect a predefined set of object categories with high accuracy, and the number of categories in object detection has grown over 10K to 100K with the help of domain adaptation and hashing\cite{Karpathy2014Deep}. However, in practical application scenarios, instead of using a predefined fixed set of object categories, one would often prefer to refer to an object with natural language rather than use a predefined category label. Such natural language query can include different types of phrases such as categories, attributes, spatial configurations and interactions with other objects, such as ��the young lady in a white dress sitting on the left�� or ��white car on the right�� in Fig.~\ref{1}.
\begin{figure}
\begin{center}
  % Requires \usepackage{graphicx}
  \includegraphics[height=10cm,width=8cm]{1.png}\\
  \caption{Overview of our method. Given an input image, a text query and a set of candidate locations (e.g. from object proposal methods), a recurrent neural network model is used to score candidate locations based on local descriptors, spatial configurations and global context. The highest scoring candidate is retrieved. }\label{1}
\end{center}
\end{figure}
In this paper, the author address the problem of natural language object retrieval: given an image and a natural language description of an object as query, they want to retrieve the object by localizing the object in the image\cite{Bansal2014What}. Natural language object retrieval can be seen as a generalization of generic object detection and has a wide range of applications, such as handling natural language commands in robotics where the user may ask to a robot to pick up ``the TV remote control on the shelf''.
\section{The model}
Inspired by the architecture of LRCN , the Spatial Context Recurrent ConvNet (SCRC) model for natural language object retrieval consists of several components as illustrated in Fig.~\ref{2}.
\begin{figure*}
\begin{center}
  % Requires \usepackage{graphicx}
  \includegraphics[height=7cm,width=18cm]{2.png}\\
  \caption{The Spatial Context Recurrent ConvNet (SCRC) for natural language object retrieval. The recurrent network in the author's model contains three LSTM units\cite{Girshick2015Fast}. Two CNN��s are used to extract local image descriptors and global scene-level contextual feature respectively. Parameters in word embedding, word prediction and three LSTM units are initialized by pretraining on image captioning dataset.}\label{2}
\end{center}
\end{figure*}
\par
At test time, given an input image $I$, a query text $S$ and a set of candidate bounding boxes $\{b_i\}$, the query text $S$ is scored on $i$-th candidate box using the likelihood of $S$ conditioned on the local image region, the whole image and the spatial configuration of the box, computed as
\begin{equation}
\begin{aligned}
s &= p(S|I_{box},I_{im},x_{spatial} \\
&=\prod_{w_t\in S}p(w_t|w_{t-1},\cdot\cdot\cdot,w_1,I_{box},I_{im},x_{spatial}).
\end{aligned}
\end{equation}
and the highest scoring candidate boxes are retrieved.
\begin{equation}
\begin{aligned}
p(w_{t+1}|w_t,\cdot\cdot\cdot,w_1,I_{box},I_{im},x_{spatial}) \\
= {\rm Softmax}(W_{local}h_{local}^{(t)}+W_{global}h_{global}^{(t)}+r)
\end{aligned}
\end{equation}
where $W_{local}$ and $W_{global}$ are weight matrices for word prediction and $r$ is a bias vector. Softmax($\cdot$) is a softmax function over a vector to output a probability distribution.
\section{Experiments}
\begin{table}[H]
\small
\renewcommand\arraystretch{1.2}
\begin{tabular}{|l|c|c|}
\hline
\textbf{Method} & P@1-NR & P@1 \\
\hline
CAFFE-7K & 32.53\% & 27.73\% \\
LRCN [4]  & - & 38.38\% \\
SCRC(w/o context, spatial, transfer) & - & 61.03\% \\
SCRC(w/o context, spatial) & - & 64.09\% \\
SCRC(w/o context) & - & 70.15\% \\
SCRC & - & \textbf{72.74\%} \\
\hline
\end{tabular} \\
\caption{Top-1 precision of our method compared with baselines on annotated bounding boxes in ReferIt dataset.} \label{tab1}
\end{table}
\par
In Table~\ref{tab1}, it can be seen that pretraining on image captioning, adding spatial configuration, and adding scene-level context all improve the performance, with adding spatial configuration $x_{spatial}$ leading to the most significant performance boost.
{\small
\bibliographystyle{ieee}
\bibliography{1}
}

\end{document}
