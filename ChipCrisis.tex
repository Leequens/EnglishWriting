\documentclass{article}
\usepackage{ctex}
\usepackage{multicol}
\usepackage[top=1in, bottom=1in, left=1.25in, right=1.25in]{geometry}
\usepackage{lscape}
\author{Zhang, Liqiang}
\date{April 21,2018}
\title{Chip Crisis}

\begin{document}

\maketitle
\par
On April 16, the U.S. Department of Commerce stated that it will prohibit US companies from selling components, goods, software and technology to ZTE for 7 years. ZTE is a victim of the escalating trade war between China and the US, but this incident also made people reexamine the development of Chines electronics industry.
\par
The chip is regarded as the "engine" of the information era and is a comprehensive embodiment of a country's high-end manufacturing capabilities. But China is currently big enough in the electronics industry but not strong enough, At present, Chinese chip demand accounts for more than 50 percent of the world's total, while domestic brand chips can only self-supplied by about 8 percent and are concentrated in the low-end market. This is a long-standing problem for the Chinese electronics industry, despite the country's strong support, research progress has been slow and the chip crisis has always existed. Actually, in many other key areas of manufacturing in China, they seem to be powerful but actually have the same risks. At present, China is undergoing a critical period of transformation and mastering the core technologies is a top priority.
\par
In my opinion, this chip crisis is not a bad thing and this may become a turning point in the development of China's electronics industry. Under such harsh conditions, China was able to develop atomic bombs and hydrogen bombs, I also believe that China can certainly have its own Chinese chip in the future. 

\end{document}
