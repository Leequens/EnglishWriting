\documentclass{article}
\usepackage{ctex}
\usepackage{multicol}
\usepackage[top=1in, bottom=1in, left=1.25in, right=1.25in]{geometry}
\usepackage{lscape}
\author{Zhang, Liqiang}
\date{April 17,2018}
\title{Vehicle Communication}

\begin{document}

\maketitle
\par
The Tesla's cash some weeks ago let people have been fear of autopilot. In the fact, the biggest problem of autopilot is how to project a system to make the car observe and judge road conditions and do some response. But in the 4G era, it is great difficult for autonomous vehicles to deal with so mach information at a very short time.  
\par
It's a  good news that 5G era is coming soon, which not only have benefits at mobile communication but also have a great help in smart home and autonomous vehicles. Yesterday, Toyota said that it will start equipping models with technology to talk to other vehicles starting in 2021 because it tries to push safety communications forward. Most of its American models should have the feature by the mid-2020s, it said. If vehicle communication really realized, the safety factor of automatic driving will increase dramatically because one car can warn another of its trend or other car's situation around it. Andrew Coetzee, vice president of product planning, says the cars would use dedicated airwaves to send signals up to nearly 1,000 feet.
\par
At present, it is unlikely that autopilot will be fully realized, because the limiting factor is not only the technical problems of automotive sensors, but also many other factors such as network technology. The current car autopilot can only be used as an auxiliary function.

\end{document}