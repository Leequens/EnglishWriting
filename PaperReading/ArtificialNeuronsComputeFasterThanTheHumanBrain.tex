\documentclass{article}
\usepackage{ctex}
\usepackage{float}
\usepackage{graphicx}
\usepackage{multicol}
\usepackage[top=1in, bottom=1in, left=1.25in, right=1.25in]{geometry}
\usepackage{lscape}
\usepackage{cite}
\author{Zhang, Liqiang}
\date{April 27,2018}
\title{Artificial Neurons Compute Faster than The Human Brain}

\begin{document}
\maketitle
\par
Nowadays, Artificial intelligence software has increasingly begun to imitate the brain. But because conventional computer hardware was not designed to run brain-like algorithms, these machine-learning tasks require orders of magnitude more computing power than the human brain does.
\begin{figure}[H]
  \centering
  % Requires \usepackage{graphicx}
  \includegraphics[height=8cm,width=12cm]{Neurons.jpg}\\
  \caption{Neurons store and transmit information in the brain}\label{Figure 1}
\end{figure}

\par
Superconducting computing chips modelled after neurons can process information faster and more efficiently than the human brain, which is described in Science Advances o, is a key benchmark in the development of advanced computing devices designed to mimic biological systems. Schneider��s team created neuron-like electrodes out of niobium superconductors, which conduct electricity without resistance. They filled the gaps between the superconductors with thousands of nanoclusters of magnetic manganese. But the issue is that the synapses can only operate at temperatures close to absolute zero, so this might make the chips impractical for use in small devices.
\par
I think that neural network research is mainly applied to academics, and large-scale commercial applications may need some time. But it is worth developing as many different technological approaches as possible, even as neuroscientists struggle to understand the human brain.
\end{document}


