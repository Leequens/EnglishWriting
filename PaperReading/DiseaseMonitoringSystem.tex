\documentclass{article}
\usepackage{ctex}
\usepackage{multicol}
\usepackage[top=1in, bottom=1in, left=1.25in, right=1.25in]{geometry}
\usepackage{lscape}
\author{Zhang, Liqiang}
\date{April 19,2018}
\title{Disease Monitoring System}

\begin{document}

\maketitle
\par
If your smartphone's camera, microphone and motion sensors were monitoring you for signs of illness, you will know about your healthy condition before the illness attack. That's the future envisioned by scientists at the Pentagon's secretive weapons development arm, where such a system is being built to keep tabs on deployed U.S. service members.
\par
The Defense Advanced Research Projects Agency announced last week that it has awarded a 5.1-million dollars contract to Kryptowire, which is a cybersecurity company. Tom Karygiannis, Kryptowire's vice president of product, said he hopes the technology broaden access to healthcare by spotting health problems before a person visits a doctor or nurse. He think that his could mean better treatment, cost savings and making treatment available to more people. But privacy advocates against this system because they think that people don't want to feel like someone is listening in on their private life and that's going to have to be subject to tight controls. However, for DARPA, the purpose is to help military to face with some healthy problem and conserve resources.
\par
In my view, if it is possible to develop a system based on this system that only sends monitoring data to users, it is not a bad idea. After all, the incubation period for many diseases is still very long. It is a great thing to use high technology to help people monitor the disease.

\end{document}
