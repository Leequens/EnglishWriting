\documentclass[10pt,twocolumn,letterpaper]{article}
\usepackage{cvpr}
\usepackage{enumerate}
\usepackage{times}
\usepackage{epsfig}
\usepackage{float}
\usepackage{graphicx}
\usepackage{amsmath}
\usepackage{amssymb}
\usepackage{multirow}
\usepackage{array}
\usepackage{cite}
\usepackage{calc}
\usepackage[pagebackref=true,breaklinks=true,letterpaper=true,colorlinks,bookmarks=false]{hyperref}
\title{\textbf{Multiscale Combinatorial Grouping}}
\author{Zhang, Liqiang\\\\June 16, 2018}
\cvprfinalcopy
\def\cvprPaperID{****}
\def\httilde{\mbox{\tt\raisebox{-.5ex}{\symbol{126}}}}
\begin{document}
\maketitle
\par
\begin{abstract}
  This paper propose a unified approach for bottom-up hierarchical image segmentation and object candidate generation for recognition, called Multiscale Combinatorial Grouping (MCG). For this purpose, the author first develop a fast normalized cuts algorithm. They then propose a high-performance hierarchical segmenter that makes effective use of multiscale information. Finally, the author propose a grouping strategy that combines their multiscale regions into highly-accurate object candidates by exploring efficiently their combinatorial space.
\end{abstract}
\section{Introduction}
Two paradigms have shaped the field of object recognition in the last decade. The first one, popularized by the Viola-Jones face detection algorithm, formulates object localization as window classification. The basic scanning-window architecture, relying on histograms of gradients and linear support vector machines, was introduced by Dalal and Triggs in the context of pedestrian detection and is still at the core of leading object detectors on the PASCAL challenge such as Deformable Part Models\cite{Arbelaez2012Semantic}.
\par
The second paradigm relies on perceptual grouping to provide a limited number of high-quality and categoryindependent object candidates, which can then be described with richer representations and used as input to more sophisticated learning methods.
\par
In this paper, the author propose a unified approach to multiscale hierarchical segmentation and object candidate generation called Multiscale Combinatorial Grouping (MCG)\cite{Hoiem2011Recovering}. Fig.~\ref{1} shows an example of the results. Their main contributions are:
\begin{figure}
\begin{center}
  % Requires \usepackage{graphicx}
  \includegraphics[height=5cm,width=8cm]{1.png}\\
  \caption{\textbf{Top:} original image, instance-level groundtruth from PASCAL and our multiscale hierarchical segmentation. \textbf{Bottom:} our best object candidates among 400.}\label{1}
\end{center}
\end{figure}
\begin{itemize}
\item An efficient normalized cuts algorithm, which in practice provides a 20$\times$ speed-up to the eigenvector computation required for contour globalization.
\item A state-of-the-art hierarchical segmenter that leverages multiscale information.
\item A grouping algorithm that produces accurate object candidates by efficiently exploring the combinatorial space of their multiscale regions.
\end{itemize}
\section{The Segmentation Algorithm}
Spatially transforming an UCM is nontrivial because its boundaries are one-dimensional entities whose topology and strength determine the underlying hierarchy\cite{David2013SCALPEL}, and an error at a single pixel can have drastic effects. The author therefore opt for sampling uniformly $K$ levels in the hierarchy, transforming them sequentially, and reconstructing from their boundaries a transformed UCM.
\par
They consider two different segmentations $\mathcal{R}=\{R_i\}_i$ and $\mathcal{S}=\{S_j\}_j$. They define the projection of the segmentation $\mathcal{R}$ onto a region $S_j\in\mathcal{S}$ as the majority label
\begin{equation}
\pi(\mathcal{R},S_j)={\rm argmax}\frac{|S_j\cap R_i|}{|S_j|}
\end{equation}
 And the projection of $\mathcal{R}$ onto $\mathcal{S}$ as
\begin{equation}
\pi(\mathcal{R},S)=\{\pi(\mathcal{R},S_j)\}_j.
\end{equation}
\par
In order to project an UCM onto a target segmentation $\mathcal{S}$, which the author denote $\pi({\rm UCM},\mathcal{S})$, they project each of the levels in the hierarchy in turn.
\begin{table*}[htbp]
\scriptsize
\renewcommand\arraystretch{1.2}
\begin{tabular}{|c|p{8.5pt}<{\centering}p{8.5pt}<{\centering}p{8.5pt}<{\centering}p{8.5pt}<{\centering}p{8.5pt}<{\centering}p{8.5pt}<{\centering}p{8.5pt}<{\centering}p{8.5pt}<{\centering}p{8.5pt}<{\centering}p{8.5pt}<{\centering}p{8.5pt}<{\centering}p{8.5pt}<{\centering}p{8.5pt}<{\centering}p{8.5pt}<{\centering}p{8.5pt}<{\centering}p{8.5pt}<{\centering}p{8.5pt}<{\centering}p{8.5pt}<{\centering}p{8.5pt}<{\centering}p{8.5pt}<{\centering}|p{8.5pt}<{\centering}|}
\hline
\textbf{VOC 2007 test}  & aero & bike & bird & boat & bottle & bus & car & cat & chair & cow & table & dog & horse & mbike & person & plant & sheep & sofa & train & tv & mAP \\
\hline
R-CNN pool$_5$ & 51.8 & 60.2 & 36.4 & 27.8 & 23.2 & 52.8 & 60.6 & 49.2 & 18.3 & 47.8 & 44.3 & 40.8 & 56.6 & 58.7 & 42.4 & 23.4 & 46.1 & 36.7 & 51.3 & 55.7 & 44.2 \\
R-CNN fc$_6$ & 43.6 & 50.4 & 32.2 & 26.0 & 9.8 & 58.5 & 50.4 & 30.9 & 7.9 & 36.1 & 18.2 & 31.7 & 41.4 & 52.6 & 8.8 & 14.0 & 26.3 & 34.0 & 53.1 & 58.0 & 46.2 \\
R-CNN fc$_7$ & 57.6 & 57.9 & 38.5 & 31.8 & 23.7 & 51.2 & 58.9 & 51.4 & 20.0 & 50.5 & 40.9 & 46.0 & 51.6 & 55.9 & 43.3 & 23.3 & 48.1 & 35.3 & 51.0 & 57.4 & 44.7 \\
\hline
R-CNN FT pool$_5$ & 57.1 & 52.0 & 31.5 & 7.6 & 11.5 & 55.0 & 53.1 & 34.1 & 1.7 & 33.1 & 49.2 & 42.0 & 47.3 & 56.6 & 15.3 &12.8 & 14.6 & 40.6 & 53.1 & 56.4 & 47.3 \\
R-CNN FT fc$_6$ & 53.1 & 57.1 & 32.4 & 12.3 & 15.8 & 58.2 & 56.7 & 39.6 & 0.9 & 44.8 & 39.9 & 31.0 & 54.0 & 62.4 & 4.5 & 20.6 & 34.2 & 50.0 & 57.7 & 63.0 & 53.1 \\
R-CNN FT fc$_7$ & 64.2 & 69.7 & 50.0 & 41.9 & 32.0 & 62.6 & 71.0 & 60.7 & 32.7 & 58.5 & 46.5 & 56.1 & 60.6 & 66.8 & 54.2 & 31.5 & 52.8 & 48.9 & 57.9 & 64.7 & 54.2 \\
\hline
R-CNN FT fc$_7$ BB & \textbf{68.1} & \textbf{72.8} & \textbf{56.8} & \textbf{43.0} & \textbf{36.8} & \textbf{66.3} & \textbf{74.2} & \textbf{67.6} & \textbf{34.4} & \textbf{63.5} & \textbf{54.5} & \textbf{61.2} & \textbf{69.1} & \textbf{68.6} & \textbf{58.7} & \textbf{33.4} & \textbf{62.9} & \textbf{51.1} & \textbf{62.5} & \textbf{64.8} & \textbf{58.5} \\
\hline
\end{tabular} \\
\caption{Detection average precision (\%) on VOC 2007 test. Rows 1-3 show R-CNN performance without fine-tuning. Rows 4-6 show results for the CNN pre-trained on ILSVRC 2012 and then fine-tuned (FT) on VOC 2007 trainval. Row 7 includes a simple bounding box regression (BB) stage that reduces localization errors. The first uses only HOG, while the next two use different feature learning approaches to augment or replace HOG.}
\label{tab1}
\end{table*}
\par
In the next section, we will iterate this procedure, and project an UCM recursively to a set of target segmentations $\{\mathcal{S}^{1\ast},\dots,\mathcal{S}^{N\ast}\}$. However, note that the composition of two such projections can be written as :
\begin{equation}
\pi(\pi({\rm UCM},\mathcal{S}^1),\mathcal{S}^2)=\pi({\rm UCM},\mathcal{S}^1)\circ\pi(\mathcal{S}^1,\mathcal{S}^2).
\end{equation}
In practice, this property means that successive projections of the target segmentations can be pre-computed, the UCM has to be projected only to the first target segmentation, and its final labels are obtained by $N-1$ look-ups.
\par
Table~\ref{tab1} rows 1-3 reveals that features from fc$_7$ generalize worse than features from fc$_7$. This means
that 29\%, or about 16.8 million, of the CNN��s parameters can be removed without degrading mAP. More surprising is that removing both fc$_7$ and fc$_6$ produces quite good results even though pool$_5$ features are computed using only 6\% of the CNN��s parameters.
{\small
\bibliographystyle{ieee}
\bibliography{1}
}
\end{document}
